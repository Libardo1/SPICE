\documentclass{article} % For LaTeX2e
\usepackage{nips15submit_e,times}
\usepackage{hyperref}
\usepackage{url}
\usepackage{multirow}

%\documentstyle[nips14submit_09,times,art10]{article} % For LaTeX 2.09

\title{Sequence Prediction using graphical Models }

\author{
Akshay Jain\\
Department of Computer Science\\
University of California,Irvine \\
\texttt{akshaj1@uci.edu} \\
\And
Varun Bharill \\
Department of Computer Science\\
University of California,Irvine \\
\texttt{email} \\
\And
Sai Teja \\
Department of Computer Science\\
University of California,Irvine \\
\texttt{email} \\
}

\newcommand{\fix}{\marginpar{FIX}}
\newcommand{\new}{\marginpar{NEW}}

\nipsfinalcopy % Uncomment for camera-ready version

\begin{document}

\maketitle

\begin{abstract}
Being able to guess the next element of a sequence is an important question in many fields, from natural language processing (NLP) to biology (DNA modelisation for instance). Different algorithms have been suggested in literature to handle the problem. In this paper, we explore Graphical model based algorithms and how they perform with respect to each other.
\end{abstract}

\section{Introduction}
\label{Introduction}
Sequenced data arises very frequently is real world. It appears DNA and RNA sequences, in sentence of words. Sequence can also be generated in a time series fashion, such as Stock prices, rain fall measurement. 

\section{Algorithms}
\label{Algorithms}
Other the period, various algorithms have been developed to predict the future values in the sequence data. 
\subsection{N-Gram Model}

\subsection{Hidden Markov Model}
\subsection{Other}

In this paper, we will focus mainly on N-gram models and Hidden Markov Models.

\section{N-gram Model}
\label{N-Gram Model}

\section{Hidden Markov Model}
\label{HMM}

\subsection{Learning Model Parameters}
\subsubsection{EM algorithm}
\subsubsection{Spectral Learning}

\section{EM algorithm}
\label{EM algorithm}
\section{Spectral Learning}
\label{Spectral Learning}


\section{Dataset}


\section{Reults}

\subsection{Figures}

All artwork must be neat, clean, and legible. Lines 

\begin{figure}[h]
\begin{center}
%\framebox[4.0in]{$\;$}
\fbox{\rule[-.5cm]{0cm}{4cm} \rule[-.5cm]{4cm}{0cm}}
\end{center}
\caption{Sample figure caption.}
\end{figure}

\subsection{Tables}

All tables must be centered, neat, clean and legible. Do not use hand-drawn
tables. The table number and title always appear before the table. See
Table~\ref{sample-table}.

Place one line space before the table title, one line space after the table
title, and one line space after the table. The table title must be lower case
(except for first word and proper nouns); tables are numbered consecutively.

\begin{table}[t]
\caption{N-Gram Model}
\label{N-Gram_model}
\begin{center}
\begin{tabular}{ |c|c|c|c| }
\hline
\multicolumn{1}{|c|}{\bf Data Set} &\multicolumn{1}{|c|}{\bf No Of Observations} &\multicolumn{1}{|c|}{\bf n-gram} &\multicolumn{1}{|c|}{\bf Accuracy}\\
\hline
\multirow{2}{*}{Data Set 1}& \multirow{2}{*}{1} & 3 & - \\
& & 4 & - \\
\hline
\multirow{2}{*}{Data Set 1}& \multirow{2}{*}{1} & 3 & - \\
& & 4 & - \\
\hline
\multirow{2}{*}{Data Set 1}& \multirow{2}{*}{1} & 3 & - \\
& & 4 & - \\
\hline
\multirow{2}{*}{Data Set 1}& \multirow{2}{*}{1} & 3 & - \\
& & 4 & - \\
\hline
\multirow{2}{*}{Data Set 1}& \multirow{2}{*}{1} & 3 & - \\
& & 4 & - \\
\hline
\end{tabular}
\end{center}
\end{table}


\begin{table}[t]
\caption{Hidden Markov Model using EM}
\label{HMM_EM}
\begin{center}
\begin{tabular}{ |c|c|c|c| }
\hline
\multicolumn{1}{|c|}{\bf Data Set} &\multicolumn{1}{|c|}{\bf No Of Observations} &\multicolumn{1}{|c|}{\bf n-states} &\multicolumn{1}{|c|}{\bf Accuracy}\\
\hline
\multirow{2}{*}{Data Set 1}& \multirow{2}{*}{1} & 3 & - \\
& & 4 & - \\
\hline
\multirow{2}{*}{Data Set 1}& \multirow{2}{*}{1} & 3 & - \\
& & 4 & - \\
\hline
\multirow{2}{*}{Data Set 1}& \multirow{2}{*}{1} & 3 & - \\
& & 4 & - \\
\hline
\multirow{2}{*}{Data Set 1}& \multirow{2}{*}{1} & 3 & - \\
& & 4 & - \\
\hline
\multirow{2}{*}{Data Set 1}& \multirow{2}{*}{1} & 3 & - \\
& & 4 & - \\
\hline
\end{tabular}
\end{center}
\end{table}

\begin{table}[t]
\caption{Hidden Markov Model using Spectral Learning}
\label{Spectral Learning}
\begin{center}
\begin{tabular}{ |l|l|l|l| }
\hline
\multicolumn{1}{|c|}{\bf Data Set} &\multicolumn{1}{|c|}{\bf No Of Observations} &\multicolumn{1}{|c|}{\bf n-gram} &\multicolumn{1}{|c|}{\bf Accuracy}\\
\hline
\multirow{2}{*}{Data Set 1}& \multirow{2}{*}{1} & 3 & - \\
& & 4 & - \\
\hline
\multirow{2}{*}{Data Set 1}& \multirow{2}{*}{1} & 3 & - \\
& & 4 & - \\
\hline
\multirow{2}{*}{Data Set 1}& \multirow{2}{*}{1} & 3 & - \\
& & 4 & - \\
\hline
\multirow{2}{*}{Data Set 1}& \multirow{2}{*}{1} & 3 & - \\
& & 4 & - \\
\hline
\multirow{2}{*}{Data Set 1}& \multirow{2}{*}{1} & 3 & - \\
& & 4 & - \\
\hline
\end{tabular}
\end{center}
\end{table}



\section{Final instructions}
Do not change any aspects of the formatting parameters in the style files.
In particular, do not modify the width or length of the rectangle the text
should fit into, and do not change font sizes (except perhaps in the
\textbf{References} section; see below). Please note that pages should be
numbered.

\section{Preparing PostScript or PDF files}

Please prepare PostScript or PDF files with paper size ``US Letter'', and
not, for example, ``A4''. The -t
letter option on dvips will produce US Letter files.

Fonts were the main cause of problems in the past years. Your PDF file must
only contain Type 1 or Embedded TrueType fonts. Here are a few instructions
to achieve this.

\begin{itemize}

\item You can check which fonts a PDF files uses.  In Acrobat Reader,
select the menu Files$>$Document Properties$>$Fonts and select Show All Fonts. You can
also use the program \verb+pdffonts+ which comes with \verb+xpdf+ and is
available out-of-the-box on most Linux machines.

\item The IEEE has recommendations for generating PDF files whose fonts
are also acceptable for NIPS. Please see
\url{http://www.emfield.org/icuwb2010/downloads/IEEE-PDF-SpecV32.pdf}

\item LaTeX users:

\begin{itemize}

\item Consider directly generating PDF files using \verb+pdflatex+
(especially if you are a MiKTeX user). 
PDF figures must be substituted for EPS figures, however.

\item Otherwise, please generate your PostScript and PDF files with the following commands:
\begin{verbatim} 
dvips mypaper.dvi -t letter -Ppdf -G0 -o mypaper.ps
ps2pdf mypaper.ps mypaper.pdf
\end{verbatim}

Check that the PDF files only contains Type 1 fonts. 
%For the final version, please send us both the Postscript file and
%the PDF file. 

\item xfig "patterned" shapes are implemented with 
bitmap fonts.  Use "solid" shapes instead. 
\item The \verb+\bbold+ package almost always uses bitmap
fonts.  You can try the equivalent AMS Fonts with command
\begin{verbatim}
\usepackage[psamsfonts]{amssymb}
\end{verbatim}
 or use the following workaround for reals, natural and complex: 
\begin{verbatim}
\newcommand{\RR}{I\!\!R} %real numbers
\newcommand{\Nat}{I\!\!N} %natural numbers 
\newcommand{\CC}{I\!\!\!\!C} %complex numbers
\end{verbatim}

\item Sometimes the problematic fonts are used in figures
included in LaTeX files. The ghostscript program \verb+eps2eps+ is the simplest
way to clean such figures. For black and white figures, slightly better
results can be achieved with program \verb+potrace+.
\end{itemize}
\item MSWord and Windows users (via PDF file):
\begin{itemize}
\item Install the Microsoft Save as PDF Office 2007 Add-in from
\url{http://www.microsoft.com/downloads/details.aspx?displaylang=en\&familyid=4d951911-3e7e-4ae6-b059-a2e79ed87041}
\item Select ``Save or Publish to PDF'' from the Office or File menu
\end{itemize}
\item MSWord and Mac OS X users (via PDF file):
\begin{itemize}
\item From the print menu, click the PDF drop-down box, and select ``Save
as PDF...''
\end{itemize}
\item MSWord and Windows users (via PS file):
\begin{itemize}
\item To create a new printer
on your computer, install the AdobePS printer driver and the Adobe Distiller PPD file from
\url{http://www.adobe.com/support/downloads/detail.jsp?ftpID=204} {\it Note:} You must reboot your PC after installing the
AdobePS driver for it to take effect.
\item To produce the ps file, select ``Print'' from the MS app, choose
the installed AdobePS printer, click on ``Properties'', click on ``Advanced.''
\item Set ``TrueType Font'' to be ``Download as Softfont''
\item Open the ``PostScript Options'' folder
\item Select ``PostScript Output Option'' to be ``Optimize for Portability''
\item Select ``TrueType Font Download Option'' to be ``Outline''
\item Select ``Send PostScript Error Handler'' to be ``No''
\item Click ``OK'' three times, print your file.
\item Now, use Adobe Acrobat Distiller or ps2pdf to create a PDF file from
the PS file. In Acrobat, check the option ``Embed all fonts'' if
applicable.
\end{itemize}

\end{itemize}
If your file contains Type 3 fonts or non embedded TrueType fonts, we will
ask you to fix it. 

\subsection{Margins in LaTeX}
 
Most of the margin problems come from figures positioned by hand using
\verb+\special+ or other commands. We suggest using the command
\verb+\includegraphics+
from the graphicx package. Always specify the figure width as a multiple of
the line width as in the example below using .eps graphics
\begin{verbatim}
   \usepackage[dvips]{graphicx} ... 
   \includegraphics[width=0.8\linewidth]{myfile.eps} 
\end{verbatim}
or % Apr 2009 addition
\begin{verbatim}
   \usepackage[pdftex]{graphicx} ... 
   \includegraphics[width=0.8\linewidth]{myfile.pdf} 
\end{verbatim}
for .pdf graphics. 
See section 4.4 in the graphics bundle documentation (\url{http://www.ctan.org/tex-archive/macros/latex/required/graphics/grfguide.ps}) 
 
A number of width problems arise when LaTeX cannot properly hyphenate a
line. Please give LaTeX hyphenation hints using the \verb+\-+ command.

\subsubsection*{References}

References follow the acknowledgments. Use unnumbered third level heading for
the references. Any choice of citation style is acceptable as long as you are
consistent. It is permissible to reduce the font size to `small' (9-point) 
when listing the references. {\bf Remember that this year you can use
a ninth page as long as it contains \emph{only} cited references.}

\small{
[1] Alexander, J.A. \& Mozer, M.C. (1995) Template-based algorithms
for connectionist rule extraction. In G. Tesauro, D. S. Touretzky
and T.K. Leen (eds.), {\it Advances in Neural Information Processing
Systems 7}, pp. 609-616. Cambridge, MA: MIT Press.

[2] Bower, J.M. \& Beeman, D. (1995) {\it The Book of GENESIS: Exploring
Realistic Neural Models with the GEneral NEural SImulation System.}
New York: TELOS/Springer-Verlag.

[3] Hasselmo, M.E., Schnell, E. \& Barkai, E. (1995) Dynamics of learning
and recall at excitatory recurrent synapses and cholinergic modulation
in rat hippocampal region CA3. {\it Journal of Neuroscience}
{\bf 15}(7):5249-5262.
}

\end{document}
